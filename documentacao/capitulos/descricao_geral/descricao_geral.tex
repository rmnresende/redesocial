\chapter{Descrição Geral}
A rede social tem como foco a integração entre alunos, professores, pesquisadores e egressos, podendo compartilhar algo para promover a curiosidade dos usuários, além de se caracterizar também como uma forma de acompanhar um pesquisador, professor, aluno ou pesquisa em específico. 	

Observando a frequência em que as pessoas utilizam de redes sociais para se comunicar com outras pessoas, percebeu-se que com a criação de uma rede social focada na docência, extensão e pesquisa, poderia-se apresentar um novo método de facilitar a comunicação entre os envolvidos. 


\section{Perspectivas do produto}

O sistema em questão trata-se de uma rede social voltada para o contexto e agregação da comunidade científica. Espera-se adesão do público nessa plataforma, a qual poderá promover melhores relacionamentos interpessoais em um ambiente colaborativo, em prol da criação e incorporação de novos conceitos e valor. A mesma se propõe a contribuir para a geração e divulgação de pesquisas científicas, com discussões e conversas produtivas, indo além das mídias sociais comumente difundidas e podendo vir a compartilhar e ampliar publicações e destaques acadêmicos gerados no âmbito de instituições educacionais.

Ademais, o \textit{software} oferecerá a possibilidade de interação com outros sistemas, como Facebook e Google. O intuito é que possam fazer parte de um leque de auxílio para a criação de contas, facilitando a participação dos usuários na rede social descrita. Assim, qualquer um poderá entrar com sua conta Google, por exemplo, sem necessidade de preenchimento de todo o cadastro solicitado pela rede em foco.

Para tanto, o sistema deverá rodar em um servidor Java - \textit{GlassFish} (servidor \textit{open source} de aplicação para Java EE), com no mínimo 8GB de memória principal, além de ambiente Linux. Ainda, verifica-se a necessidade de HD (Disco rígido) de pelo menos 300 GB, além de processador \textit{multicore} eficiente, em um frequência com especificações mínimas de 3.00 GHz e 4 núcleos. 


\section{Funções do produto}

O sistema terá funções como: cadastrar usuários, login, enviar solicitações de amizades, edição de perfil, novidades, postagem de publicações e fotos, além de criar, pesquisar, participar, convidar, sair e denunciar grupos, editar e gerenciar sua conta, envio de mensagens, alteração de status do bate papo e o cadastro de artigos científicos.

\section{Características dos usuários}
Os usuários serão estudantes de cursos de graduação, estudantes de pós-graduação, mestrado e doutorado, além de professores e pesquisadores, com o intuito de socialização, debates de assuntos científicos como artigos de periódicos, estabelecimento de conexões que viabilizem iniciativas de pesquisa científica e/ou de produtos, tornando-se uma maneira mais democrática de disseminar o conhecimento científico nos círculos sociais dos usuários.


\section{Restrições gerais}
O \textit{software} será desenvolvido utilizando HTML 5 e CSS3 na construção do ambiente \textit{web}. Também será usada a linguagem de programação Java, devido ao grande volume de informações que serão geradas pelos usuários. O banco de dados escolhido foi o MySQL por ser \textit{open source}, além de possuir melhor aceitação pela equipe desenvolvedora.

\section{Suposições e dependências}
O \textit{software} exige que os seguintes programas estejam previamente instalados no computador: \textit{MySQL} e \textit{GlassFish}. Também exige que a máquina do usuário tenha no mínimo processador \textit{dual core} e 8 GB de RAM para rodar perfeitamente. Funcionará em computadores \textit{desktop} e \textit{smartphones} por meio dos seguintes navegadores: Mozilla Firefox e Google Chrome. Terá suporte de funcionamento no Windows e no Linux (especificamente para distribuição Ubuntu).